The back-end use dotNet framework to develop, the language is C\#.

During the development of the back-end, the overall of development is created 
the Clean Architecture for the back-end.

For creating the architecture, we need the following layers:
\begin{itemize}
    \item Domain Layer: 
    \item Application Layer: The application layer is the layer that contains
    the application logic, which is implemented the domain service interface
    and offer the service to the presentation layer, infrastructure layer and the test layer.
    \item Infrastructure Layer: The infrastructure layer is the layer that
    contains the implementation of the application service interface, 
    there are not just one of the infrastructure but many, for example,
    \begin{itemize}
        \item Database infrastructure: The database infrastructure is the layer which
        contains the implementation of the database service interface,
        it can be the DTO, the Repository , the Query, etc.
        \item MQTT infrastructure: The MQTT infrastructure will contain the back-end MQTT client which 
        will subscribe the MQTT topic and publish the MQTT topic though the MQTT broker to communicate
        with the device.
        \item WebRTC's infrastructure: The WebRTC infrastructure is working for offer the WebRTC service
        which specific as the ICE config and TURN server.
    \end{itemize} 
    \item Test Layer: The test layer is the layer that contains the test code for all layers.
\end{itemize}

\paragraph{Domain Layer}
The domain layer is the heart of the software, 
it contains the business logic and the entities of the software.

The domain layer include two parts:

The entities of the software, for example, the user entity, the device entity, etc. 
See the figure \ref{fig:domainLayerEntities}.

\begin{figure}[htbp]
    \begin{lstlisting}[language=CSharp]
        public class Device{
            public string Id { get; set; }
            public string? SerialNumber { get; set; }
            public string? ShortName { get; set; }
            public Organization? OrganizationID  { get; set; }
            public Group? GroupID { get; set; }
        }
    \end{lstlisting}
    \caption{The Device entity}
    \label{fig:domainLayerEntities}
\end{figure}

The domain service of the software, this part included the interface 
which identify the service which the application layer should be implemented
to reach the operation of the domain. 

It uses for limited the application layer to access the domain layer. 

See the figure \ref{fig:domainLayerInterface}.
\begin{figure}[h]
    \begin{lstlisting}[language=CSharp]
        public interface IDeviceService
        {
            List<Device> GetDeviceList();
            Device GetDeviceById(int id);
            Device CreateDevice(Device device);
            Device UpdateDevice(Device device);
            Device DeleteDevice(int id);
        }
    \end{lstlisting}
    \caption{Domain Layer Interface}
    \label{fig:domainLayerInterface}
\end{figure}

\paragraph{Application Layer}
The application layer is the layer that implement the domain service interface
and offer the service to the presentation layer, infrastructure layer and the test layer.
See the figure \ref{fig:applicationLayerService}.
\begin{figure}[htbp]
    \begin{lstlisting}{language=CSharp}
    public class DeviceService: IDeviceService{

        private IDeviceRepository _deviceRepository;

        public DeviceService (IDeviceRepository deviceRepository)
        {
            _deviceRepository = deviceRepository ?? throw new InvalidDataException("deviceRepository can not be null");
        }

        public List<Device> GetAllDevice()
        {
            return _deviceRepository.FindAll();
        }

        public Device CreateDevice(Device device)
        {
            return _deviceRepository.CreateDevice(device);
        }

        public Device ReadDevice(Device device)
        {
            return _deviceRepository.ReadDevice(device);
        }

        public Device UpdateDevicee(Device device)
        {
            return _deviceRepository.UpdateDevice(device);
        }

        public Device DeleteDevice(Device device)
        {
            return _deviceRepository.DeleteDevice(device);
        }
}
    \end{lstlisting}
    \caption{Application Layer Service}
    \label{fig:applicationLayerService}
\end{figure}

\paragraph{Infrastructure Layer}
The fracture layer working for implement the external service interface.
For the overall of the fracture layer, it can be the webAPI, the database repository, the MQTT, the WebRTC, etc.

The webAPI implement the webAPI service interface from the application layer.
And the webAPI will offer the RESTful API to the presentation layer which is the webAPI controller.
Also, the DTO, the Program.cs file as the entry point of the back-end service and config file.

The database repository implements the database service interface from the application layer.
Out of the database repository, it also can be the database DTO, DbContext, Entities etc.

The MQTT implement the MQTT service interface from the application layer, which contain the MQTT client,
the MQTT broker, the MQTT topic utility, etc.

\paragraph{Test Layer}
The test layer is the layer that offer the unit test for all the layers. Which means that the test layer
reference all the layers but not necessary to implement the interface of the layers.

The unit test is testing the data type correct, the interface of the method have the correct parameters 
and the return value type is correct, also the functions and methods interact as pure function, 
which means that the function or method should not have the side effect. 

\paragraph{Dependency Injection}
The dependency injection is the design pattern which use for inject the dependency of the class.
This is the most important design pattern for the Clean Architecture.

\pagebreak