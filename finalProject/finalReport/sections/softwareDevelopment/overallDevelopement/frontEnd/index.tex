The front overall development will create the basic front-end for the project.
For developing the front-end, we need to create the following components:
\begin{itemize}
    \item pages
    \item components
    \item hooks
    \item store
    \item utils
    \item styles
\end{itemize}

\paragraph{Pages}
In the React, the page is the container of the application using the JSX syntax.
The page is the structure of the application. 

See the figure \ref{fig:devicePage} for the device page example.

\begin{figure}[htbp]
    \begin{lstlisting}[language=React]
        const devicePage= () => {
            return (
                <>
                    <headerComponent />
                    <deviceComponent />
                    <footerComponent />
                </>
            )
        }
    \end{lstlisting}
    \caption{The device page of the application}
    \label{fig:devicePage}
\end{figure}

\paragraph{Components}
The components are the reusable elements in the application.
Which included two types of components: 
\begin{itemize}
    \item page components: the reusable page components included multiple element components. 
    Such as the user component, the device component, the group component, etc.
    \item element components: the reusable basic elements.
    Such as the button, the input, the select, etc.
    \item specific components: the components reusable with the specific usage.
    Such as the session component, the data channel component, etc.
\end{itemize}

See the figure \ref{fig:deviceComponent} for the device component example.

See the figure \ref{fig:tableComponent} for the table component example.

\begin{figure}[htbp]
    \begin{lstlisting}[language=React]
        const deviceComponent = () => {
            const {
                data: deviceData, 
                loading: deviceLoading, 
                error: deviceError,
                getDeviceList
                } = useGetDevicelist();
            
            useEffect(() => {
                getDeviceList();
            }, []);

            return (
                <>
                    {deviceError && 
                        <div>Failed to load device list</div>}
                    {deviceLoading && 
                        <div>Loading...</div>}
                    {!deviceLoading && !deviceError && 
                        <TableComponent data={deviceData} />}
                </>
            )
        }
    \end{lstlisting}
    \caption{The device component of the application}
    \label{fig:deviceComponent}
\end{figure}
\begin{figure}[htbp]
    \begin{lstlisting}[language=React]
        const Table = ({ data }) => {
            const columns = useMemo(() => COLUMNS, []);
            const tableData = useMemo(() => data, [data]);
            const {
                getTableProps,
                getTableBodyProps,
                headerGroups,
                rows,
                prepareRow,
                } = useTable({
                    columns,
                    data: tableData,
                });
            
            return (
                <table {...getTableProps()}>
                    <thead>
                        {headerGroups.map(headerGroup => (
                            <tr {...headerGroup.getHeaderGroupProps()}>
                                {headerGroup.headers.map(column => (
                                    <th {...column.getHeaderProps()}>
                                        {column.render('Header')}
                                    </th>
                                ))}
                            </tr>
                        ))}
                    </thead>
                    <tbody {...getTableBodyProps()}>
                        {rows.map(row => {
                            prepareRow(row);
                            return (
                                <tr {...row.getRowProps()}>
                                    {row.cells.map(cell => {
                                        return (
                                            <td {...cell.getCellProps()}>
                                                {cell.render('Cell')}
                                            </td>
                                        )
                                    })}
                                </tr>
                            )
                        })}
                    </tbody>
                </table>
            )
        }
    \end{lstlisting}
    \caption{The Table component}
    \label{fig:tableComponent}
\end{figure}

\paragraph{Hooks}
The hooks are the reusable functions in the application which used the React customize hooks.
The hooks should be the pure functions. And should operate its own scope and state.
The hook can be invoked in any of the components. 
See the figure \ref{fig:useGetDeviceList} for the useGetDeviceList hook example.

\begin{figure}[htbp]
    \begin{lstlisting}[language=React]
        export const useGetDeviceList = () => {
            const [deviceList, setDeviceList] = useState([]);
            const [loading, setLoading] = useState(true);
            const [error, setError] = useState(null);

            const getDeviceList = async () => {
                setLoading(true);

                return fetch(
                    env + "/api/devices",
                    {
                        method: "GET",
                        headers: {
                            "Content-Type": "application/json",
                            Authorization: `Bearer ${localStorage.getItem(
                                "token"
                            )}`,
                        },
                    }
                )
                    .then((res) => res.json())
                    .then((data) => {
                        setDeviceList(data);
                        setLoading(false);
                    })
                    .catch((error) => {
                        setError(error);
                        setLoading(false);
                    });
                )
            };

            useEffect(() => {
                deviceList.length === 0 && getDeviceList();
            }, []);

            return { deviceList, loading, error };
        }
    \end{lstlisting}
    \caption{The useGetDeviceList hook}
    \label{fig:useGetDeviceList}
\end{figure}

\paragraph{Store}
The store is the global state of the application. 
In this project we use the Redux to manage the global state.
Instead, using the original Redux, we use the Redux Toolkit to simplify the Redux development.

Therefore, the store's \texttt{index.js} file is the Redux Toolkit store 
which used for combine the reducers and the middlewares. 
The store's \texttt{slice.js} file is the reducer of the store 
which used for manage the state of the store. 

See the figure \ref{fig:storeIndex} for the store index example. 

The Slice is the reducer of the store. 
It is a collection that contains the state and the reducers for the one kind of the states.

For example, the deviceSlice is the slice for the device state. 
It contains multiple states such like the deviceList, the deviceInfo, the deviceStatus, etc.
And it contains multiple reducers to modify the relevant states when there is an action to cause the state change.

See the figure \ref{fig:deviceSlice} for the deviceSlice example.
\begin{figure}[htbp]
    \begin{lstlisting}[language=React]
        import { createSlice } from "@reduxjs/toolkit";
        import { getDeviceList } from "../../api/device";

        const initialState = {
            deviceList: [],
            deviceListLoading: false,
            deviceListError: null,
        };

        const deviceSlice = createSlice({
            name: "deviceSlice",
            initialState,
            reducers: {
                getDeviceListStart(state) {
                    state.deviceListLoading = true;
                    state.deviceListError = null;
                },
                getDeviceListSuccess(state, action) {
                    state.deviceListLoading = false;
                    state.deviceList = action.payload;
                },
                getDeviceListFailure(state, action) {
                    state.deviceListLoading = false;
                    state.deviceListError = action.payload;
                },
            },
        });

        export const {
            getDeviceListStart,
            getDeviceListSuccess,
            getDeviceListFailure,
        } = deviceSlice.actions;

        export const fetchDeviceList = () => async (dispatch) => {
            try {
                dispatch(getDeviceListStart());
                const deviceList = await getDeviceList();
                dispatch(getDeviceListSuccess(deviceList));
            } catch (error) {
                dispatch(getDeviceListFailure(error));
            }
        };

        export default deviceSlice.reducer;
    \end{lstlisting}
    \caption{The deviceSlice}
    \label{fig:deviceSlice}
\end{figure}

\begin{figure}[htbp]
    \begin{lstlisting}[language=React]
        import { configureStore } from "@reduxjs/toolkit";

        const store = configureStore({
            reducer: {
                deviceSlice: deviceReducer,
            },
            middleware: (getDefaultMiddleware) => getDefaultMiddleware({
                    serializableCheck: false,
            })
        });
        export default store;
    \end{lstlisting}
    \caption{The store index}
    \label{fig:storeIndex}
\end{figure}

\paragraph{Utils}
The utils are the pure functions in the application which can be reused in any of the components.

\paragraph{Styles}
The styles are the style sheets of the application.

\pagebreak