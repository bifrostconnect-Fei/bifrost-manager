Before we start the feature development, we can identify the use case to 
explicitly define the requirements for the feature. 

The use case is defined as follows:

\begin{table}[htpb]
    \centering
    \begin{tabular}{|p{0.3\linewidth}|p{0.6\linewidth}|}
        \hline
        \textbf{Use Case 1} & 
        \textbf{The user can get device list} \\
        \hline
        \textbf{Actors} & User \\
        \hline
        \textbf{Preconditions} & 
        \begin{enumerate}
            \item The user is logged in
            \item The user is on the device page
            \item The user has at least a device online
        \end{enumerate}\\
        \hline
        \textbf{Basic Flow} & 
            The user will have the vision of the device list \\
        \hline
        \textbf{Alternative Flows} & 
        null \\
        \hline
        \textbf{Postconditions} & The device list display on the device page \\
        \hline
    \end{tabular}
    \caption{Use Case 1}
    \label{tab:useCase1}
\end{table}

Following the use case, we can start the feature development.
