% 项目的背景
% 这一部分要详细的介绍关于项目的原始背景, 因为在之后要根据这些背景写明项目的意义和解决方案.
% 所以项目的现状, 项目的历史问题, 现有项目难以开发和维护的原因, 

% The current BifrostConnect front-end application consist of the Device Manager (DM) 
% and Remote Access Interface (RAI) two different user interface.
% Which means the user need to manage their devices and users in the one application, 
% and oprate their devices to use the bifrost product in other web application as well. 
% However, the RAI consists of two different user interfaces referred to as the Classic RAI and Tunnel RAI. 
% Those two user interfaces will leading the user to different web application 
% when user wish to create a different type of access solution.

% The reason for this situation occurred is because the bifrost product is developed focus on the device 
% and the application part was not considered as important as the device part. 
% Therefore, the application part was developed by different developers with different design logic.
% And the application functionality was not been properly through out the development process. 
% Which cause the design logic of this product is not scalable, leading to decreasing user experience, 
% and exponential growth of code complexity which makes it increasingly challenging to maintain and improve. 
% Also the product design style lacks scalability due to lack of modularity and code consistency, 
% and the operation logic is not clear.

The existing BifrostConnect front-end application is divided into two distinct user interfaces:
the Device Manager (DM) and the Remote Access Interface (RAI). 
This division requires users to manage their devices and users in one application, 
while operating their devices to use the Bifrost product in another web application. 
Moreover, the RAI is further split into two different user interfaces, 
known as the Classic RAI and Tunnel RAI. 
These interfaces direct users to different web applications 
depending on the type of access solution they wish to create.

This situation arose because the development of the Bifrost product was primarily focused 
on the device, with the application part not receiving as much attention. 
As a result, the application part was developed by different developers, 
each with their own design logic. 
This led to a lack of thorough planning for the application functionality 
during the development process. Consequently, 
the design logic of the product is not scalable, 
which has resulted in a diminished user experience 
and an exponential increase in code complexity, 
making it increasingly difficult to maintain and improve. 
Furthermore, the product design style lacks scalability due to a lack of modularity 
and code consistency, and the operation logic is unclear.