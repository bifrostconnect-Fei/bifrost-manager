The existing BifrostConnect front-end application is divided into two distinct user interfaces:
the Device Manager (DM) and the Remote Access Interface (RAI). 
This division requires users to manage their devices and users in one application, 
while operating their devices to use the Bifrost product in another web application. 
Moreover, the RAI is further split into two different user interfaces, 
known as the Classic RAI and Tunnel RAI. 
These interfaces direct users to different web applications 
depending on the type of access solution they wish to create.

This situation arose because the development of the Bifrost product was primarily focused 
on the device, with the application part not receiving as much attention. 
As a result, the application part was developed by different developers, 
each with their own design logic. 
This led to a lack of thorough planning for the application functionality 
during the development process. Consequently, 
the design logic of the product is not scalable, 
which has resulted in a diminished user experience 
and an exponential increase in code complexity, 
making it increasingly difficult to maintain and improve. 
Furthermore, the product design style lacks scalability due to a lack of modularity 
and code consistency, and the operation logic is unclear.
