% Overall Architecture Design

The overall architecture design outlines the comprehensive structure of the system. 
The primary objective of this design is to ensure the system's functionality, 
reliability, scalability, and maintainability.

In this project, the architecture design is based on the Client/Server (C/S) model, 
with an additional module incorporated for facilitating remote communication.

The client-side of the application is a web browser, serving as the primary interface 
for user interaction. User operations include data requests, information display, 
data modification, device configuration, and the creation 
and management of classic remote access and tunnels.

The server-side of the application is responsible for processing client requests, 
executing database operations (such as Create, Read, Update, and Delete operations) 
based on the request content, and returning the results to the client.

The additional module in the back-end for remote communication is designed to implement 
the functionality of remote communication, ensuring its reliability, scalability, 
and maintainability. 

In this context, the MQTT protocol is employed, 
with the open-source Mosquitto serving as the MQTT server 
and the open-source Paho as the MQTT client. 

The WebRTC, an open-source technology, 
is utilized for remote communication.

The system utilizes a Relational Database Management System (RDBMS), 
specifically SQLite, for data management.

In the C/S architecture, the RESTful API serves as the interface definition, 
JSON is used as the data transmission format, 
and HTTP is used as the data transmission protocol.
Between the front-end with the MQTT broker and WebRTC server, the FlatBuffers protocol
will be implemented into this data transmissions. 

And there are an end point which is the device will also create a connection together
with the Front-end and Back-end.