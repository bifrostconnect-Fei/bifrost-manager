% 整体架构设计

% 整体架构设计讨论了整个系统的总体结构, 包括系统的总体目标, 总体原则, 总体设计思路, 总体模块划分, 模块之间的关系, 模块的功能和作用, 模块的接口定义, 模块的实现说明等.
% 整体架构的设计目标是为了实现整个系统的功能, 保证系统的可靠性, 可扩展性, 可维护, 使用 C/S 架构, 
% 附加了一个创建远程通信的模块.

In this project, the overall architecture design is based on C/S architecture, 
The client is a web browser. All the user interaction is done through the web browser.
The uses operation include require data, display infomation, modify data, 
configure the device, create and oprating a classic remote access and tunnel will be happen
in the web browser which is the frontEnd of the system.

% 在 C/S 架构中, 客户端是一个 web 浏览器, 所有的用户交互都是通过 web 浏览器完成的, 
% 包括数据的请求, 信息的显示, 数据的修改, 设备的配置, 创建和操作经典远程访问和隧道等.

% 服务端是一个服务器, 服务器的功能是接收客户端的请求, 并且根据请求的内容, 对数据库进行增删改查的操作,
% 并且将操作的结果返回给客户端.

% 数据库使用的是 Relational Database Management System (RDBMS), 采用简单轻量的 SQLite.

% 在C/S架构中, 采用了 RESTful API 作为接口定义, 采用了 JSON 作为数据的传输格式, 采用了 HTTP 作为数据的传输协议.

% 附加的创建远程通信的模块, 是为了实现远程通信的功能, 保证远程通信的可靠性, 可扩展性, 可维护性,
% 在这个原则上使用了 MQTT 协议, 采用了开源的 Mosquitto 作为 MQTT 服务器, 采用了开源的 Paho 作为 MQTT 客户端.
% 采用了开源的 WebRTC 作为远程通信的技术.



