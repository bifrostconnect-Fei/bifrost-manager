% 后端架构设计
% 后端使用的是 dotNet 框架.
% 在 dotNet 框架中, 使用 C\# 语言进行开发. 采用了洋葱架构,
% 洋葱架构是一种典型的分层架构, 每一层都依赖于内层, 内层向外层提供接口, 
% 外层根据内层的接口, 向更外层和同一层的其他模块提供服务.
%
% 从内到外分别是: Domain, Application, Infrastructure, Presentation.
% 其中 Domain 层是核心层, 也是最内层, 用于定义领域模型, 也就是业务模型.
%   其中包括了领域模型, 领域服务接口
%   领域模型是用于定义业务模型的, 也就是实体关系模型中的实体以及它们的属性.
%   领域服务接口是用于定义业务服务的, 也就是实体关系模型中的实体之间的关系.
%       
% Application 层是应用层, 用于定义应用服务, 也就是业务逻辑.
%   其中包括了领域服务实现和应用服务接口.
%   领域服务实现实现了内层领域服务接口的方法, 实现了领域模型的业务逻辑.
%   应用服务接口是用于定义应用服务的, 也就是业务逻辑. 
%   其中包括但不限于数据库接口, 测试接口, httpApi接口, mqtt接口等.
%   
% Infrastructure 层是基础设施层, 用于定义基础设施
%   其中包括了数据库实现, 测试实现, httpApi实现, mqtt实现等.
%   数据库实现实现了数据库接口, 实现了数据库的增删改查服务.
%   测试实现实现了测试接口, 实现了单元测试和集成测试的相关服务.
%   httpApi实现实现了httpApi接口, 实现了httpApi的增删改查.
%   mqtt实现实现了mqtt接口, 实现了mqtt的增删改查.
%
% Presentation 层是展示层, 用于定义展示逻辑, 也就是接口, 页面等. 由于这里是后端项目, 
% 数据的展示和控制是由前端完成的, 所以这一层是不需要的.
