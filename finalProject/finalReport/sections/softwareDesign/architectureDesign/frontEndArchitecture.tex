% 在前端的架构设计中
% 设计的目标是基于用户的使用需求, 也就是能够直观的对数据进行增删改查的操作, 以及直观简单的创建数据通道,
% 以及对数据通道进行管理, 以及对数据通道进行数据的可视化展示.
% 因此, 在前端中使用了基于视图设计的 React 框架, 以及基于状态和数据流的 Redux 框架.

% React 框架的结构, 根据模块化的需要, 将 web 页面分为了多个组件, 每个组件都有自己的状态和属性,
% 组件分为了容器组件, 展示组件和功能组件, 容器组件负责管理数据和状态, 展示组件负责展示数据和状态, 功能组件负责实现功能.

% 在本项目中, 根据 React 的框架结构, 使用组件化的思想,
% 将整个项目划分为以下组件:
% 1. 页面 pages : 负责整体页面展示的框架, 包括页面的路由功能.
% 2. 组件 compoents: 负责页面各个组成部分, 以及它们的状态, 数据和功能.
% 3. store: 负责整个前端项目的公共数据的管理以及更新.
% 4. utils: 包括了整个项目公用的纯函数.
% 5. hooks: 包括了整个项目公用的自定义 hooks. 其中包括了对数据的请求, 以及对数据的处理.




\paragraph{Front-end architecture}
\paragraph{Front-end data flow} 