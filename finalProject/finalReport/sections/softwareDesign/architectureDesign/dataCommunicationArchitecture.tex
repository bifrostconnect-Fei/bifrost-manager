% 关于本项目的数据通信与数据存储的设计, 包括数据通信的协议, 数据存储的设计等
% 关于数据通信的设计:
% 1. 通信协议的选择
% 自前端向后端发送的数据, 有三种传输的数据类型, 
% 一种是普通的增删改查的请求, 对数据传输的时效性要求不高, 但是对数据的准确性, 完整性, 有序性, 安全性有一定的要求,
% 这种数据的传输, 采用 HTTP 协议, 以及 RESTful API 的设计. 可以有效的保证对数据传输的以上要求.
% 一种是对数据通道的创建和流媒体数据的传输, 对数据传输的时效性, 安全性要求较高, 这种数据的传输, 采用 WebRTC 协议, 以及 MQTT 协议.
% 配合可以快速解码的 flatbuffers 协议, 可以有效的保证对数据传输的以上要求.
% 最后一种是对设备的状态信息和操作信息的传输, 对完整性, 有序性, 安全性都有较高的要求, 这种数据的传输, 采用 MQTT 协议
% 同时也使用了 flatbuffers 协议.
% 
% 2. 数据通信的通信架构和通信流程
% 本项目的数据通信的通信架构, 是基于前后端分离的架构, 前端使用 React 框架, 后端使用 dotnet 框架.
% 当前端需要向后端发送数据的时候, 前端会向后端发送 HTTP 请求, 后端接收到 HTTP 请求之后, 会根据请求的数据类型,
% 选择不同的数据处理方式, 对于普通的增删改查的请求, 后端会根据 RESTful API 的设计, 对数据进行增删改查的操作,
% 对于对数据通道的创建和流媒体数据的传输, 后端会根据 WebRTC 协议, 对数据通道进行创建, 并且帮助前端和设备建立数据通道,
% 当数据通道建立后, 前端和设备之间则使用 flatbuffer 的数据格式对流媒体数据进行传输,
% 对于设备的状态信息和操作信息的传输, 前端会直接向 MQTT broker 发送 MQTT 请求, 
% 设备会在其自身的固件中监听相关的 MQTT 请求, 并且返回相关的数据.
% 
% 3. 数据通信的格式
% 本项目的数据通信的格式, 有三种, 
% 一种是 HTTP 协议, 
% 使用 json 格式对数据进行传输,
% 一种是 WebRTC 协议, 
% 使用 flatbuffers 格式对数据进行传输,
% 一种是 MQTT 协议.
% 使用 flatbuffers 格式对数据进行传输,
% 
% 关于数据存储的设计:
% 1. 数据存储的数据库的选择
% 本项目的数据存储的数据库的选择, 使用了轻量级的数据库 SQLite,
% SQLite 是一个进程内的库, 实现了自给自足的, 无服务器的, 零配置的, 事务性的 SQL 数据库引擎.
% 这是因为整个项目的目的是为了实现前端与设备之间的数据通信, 对于数据库数据的增删改查操作的要求不高,
% 数据量较小, 且对于数据库的数据的事务性要求不高, 所以选择了 SQLite 数据库.
% 2. 项目前后端的数据结构的设计
% 在本项目中, 前端由于使用了 React 框架, 所以前端的数据结构的设计, 使用了基于状态的数据结构的设计,
% 每个组件或者数据集都包含一个状态对象, 这个状态对象的属性就是组件的各个状态. 
% 使用状态对象的原因是, 可以方便的对状态进行管理, 采用对象-属性的形式, 可以方便的针对不同组件的同类状态进行区分,
% 由于跨组件的状态是由 redux 进行管理的, 这种状态对象的设计, 可以更搞笑的对状态进行更新和传递.
% 后端由于使用了 dotnet 框架, 所以后端的数据结构的设计, 使用了基于类的数据结构的设计,
% 采用了面向对象的编程思想, 对数据进行了封装, 使得数据的传输更加的安全, 有序, 完整.

