Once the software is built, it is delivery to the test environment. 
The test environment is a server that is used to E2E test and performance test the software.

In this project, the test environment is a server that is running on the docker container.
Therefore, the GitHub Actions will build the docker image and push the image to the private docker registry.

After the docker image is pushed to the private docker registry, 
the docker image will be pulled by the test environment server.
Then the docker image will be run on the test environment server 
with the fully project including the front-end, back-end, database, MQTT broker/client, and the WebRTC ICE server/ TUNE server.

\paragraph{E2E Testing}

The E2E testing is the practice of testing the software from the user's perspective.
The E2E testing will test the software as a whole project, 
which means the front-end, back-end, database, MQTT broker/client, and the WebRTC ICE server/ TUNE server.

The E2E testing start from the front-end, the E2E testing framework will open the browser in the test environment server,
and then the E2E testing framework will simulate the user's action to require the feature's function.
And then check the result of the action is correct or not.

For example, the E2E testing framework will open the browser and then login to the system.
After login to the system, the E2E testing framework will check the login result as the expected result.

\paragraph{Performance Testing}

The performance testing is the practice of testing the software's performance.
It's including the response time, throughput, and the resource utilization.


