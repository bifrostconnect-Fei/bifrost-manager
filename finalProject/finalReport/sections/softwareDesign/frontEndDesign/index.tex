% 前端的设计, 由于已经存在了前端的之前的原始代码, 而且本项目是集成两个界面
% 到已经存在的前端代码中, 所以只需要简单介绍前端的结构.
% 然后设计新的界面, 以及功能逻辑, 再说明如何集成到原有的前端代码中.

% 集成 RAI

% 前端界面设计

% 前端的界面结构, 是根据原有的界面结构的样式进行修改, 使其更加符合响应式页面的要求.
% 响应式页面的要求是, 页面的布局, 随着屏幕的大小而变化, 使得在不同的屏幕上, 页面
% 的布局都能够合理的显示, 从而使得用户能够更加方便的使用页面.

% 根据以上的要求, 我们定义了屏幕尺寸的适配:
% 迷你屏幕 0px - 575px
% 小屏幕 576px - 767px
% 中等屏幕 768px - 991px
% 大屏幕 992px - 1199px
% 超大屏幕 1200px - 无穷大

% 为了适配不同的屏幕, 我们使用栅格系统, 使得页面的布局能够随着屏幕的大小而变化.
% 页面样式的实现使用的是 tailwindcss, 一个基于 utility-first 的 css 框架.
% 通过使用 tailwindcss, 可以将页面样式集成到页面结构中去, 从而便于实现模块化的页面组件

% 前端的架构设计, 使用了 react 的设计思想, 将页面分为多个组件, 每个组件都有自己的
% 页面结构和样式, 从而使得页面的结构更加清晰, 便于维护和修改.

% 