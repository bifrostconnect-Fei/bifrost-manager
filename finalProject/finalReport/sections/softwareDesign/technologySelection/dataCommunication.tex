\paragraph{HTTP}
is an application-layer protocol for transmitting hypermedia documents, such as HTML.
It was designed for communication between web browsers and web servers, but it can also be used for other purposes.
HTTP follows a classical client-server model, with a client opening a connection to make a request,
then waiting until it receives a response.
HTTP is a stateless protocol, meaning that the server does not keep any data (state) between two requests.
Though often based on a TCP/IP layer, it can be used on any reliable transport layer;
that is, a protocol that doesn't lose messages silently, such as UDP.

\paragraph{Flatbuffers}
is an efficient cross platform serialization library for C++, C\#, C, Go, Java, JavaScript, Lobster, Lua, TypeScript, PHP, Python, and Rust.
It was originally created at Google for game development and other performance-critical applications.

\paragraph{MQTT}
is a machine-to-machine (M2M)/"Internet of Things" connectivity protocol.
It was designed as an extremely lightweight publish/subscribe messaging transport.
It is useful for connections with remote locations where a small code footprint is required and/or network bandwidth is at a premium.

\paragraph{webrtc}
is a free, open-source project that provides web browsers and mobile applications with real-time communication (RTC) via simple application programming interfaces (APIs).
It allows audio and video communication to work inside web pages by allowing direct peer-to-peer communication,
eliminating the need to install plugins or download native apps.

\paragraph{WebRTC Datachannel}
is a component of WebRTC that enables peer-to-peer exchange of arbitrary data between two devices.
It was added to the WebRTC standard by the World Wide Web Consortium (W3C) in January 2015.
