% As we already discrbed about the bussines target and requirement 
% about this project. We can identify the bussines domain model as following
% entitis:

% \paragraph{Organization}
% The organization have all the users and devices, which can set up the users and devices as a group.
% Therefore the organization have the following attributes:
% \begin{itemize}
%     \item name: The name of the organization.
%     \item users: which will be a list of users.
%     \item devices: which will be a list of devices.
%     \item groups: which will be a list of groups.
% \end{itemize}

% \paragraph{User}
% The user can be the technitian or the oprator who is using the device to connect
% to the remote equipment. The user will be the end point which will start all of the process and end of the process as well.
% The user have the following attributes:
% \begin{itemize}
%     \item username: which will identify the user.
% \end{itemize}

% \paragraph{Device}
% The device is the hardware which will be connected to the remote equipment to allowed the user to control the remote equipment.
% The device have the following attributes:
% \begin{itemize}
%     \item name: which will identity the device.
% \end{itemize}
    
% \paragraph{Group}
% The group is the collection of the users and devices, which will be set up by the organization. 
% Only the user and the device belong to the same group, the user can oprate the device, and only both the devices in the same group.
% The user can create the tunnel between the devices in the same group.
% The group have the following attributes:
% \begin{itemize}
%     \item name: which will identity the group.
%     \item users: which will be a list of users.
%     \item devices: which will be a list of devices.
% \end{itemize}

% This section provides an analysis of the business domain model.
% Based on the project requirements, the following entities can be identified in the business domain model:

% User: Represents both regular users and administrator users.
% Device: Categorized into "attend device" and "non-attend device".
% Organization: Contains groups and is the owner of all devices and users.
% Group: Belongs to an organization and includes a list of users and devices.

% The logic of this section is as follows:
% First, we analyze the business domain model and then abstract it into an object domain model.
% The object domain model serves as the foundation for the entire project's domain model.
% Based on this, we can design the database model by defining tables as entities and table columns as attributes of the model.
% Additionally, we specify relationships between tables, define primary and foreign keys, and obtain the database model.
% Furthermore, we can design the domain model for backend objects, which serves as the foundation for backend objects.

Following the identified stakeholders, we can now analyze the business domain model.
Based on the project requirements, 
the following entities can be identified in the business domain model:

\paragraph{Organization}
The organization has all the users and devices and can group them together.
The organization has the following attributes:
\begin{itemize}
    \item name: The name of the organization.
    \item users: A list of users.
    \item devices: A list of devices.
    \item groups: A list of groups.
\end{itemize}

\paragraph{User}
The user can be a technician or an operator who uses the device to connect to remote equipment.
The user is the starting and ending point of all processes.
The user has the following attributes:
\begin{itemize}
    \item username: The identifier of the user.
\end{itemize}

\paragraph{Device}
The device is the hardware that connects to remote equipment, allowing the user to control it.
The device has the following attributes:
\begin{itemize}
    \item name: The identifier of the device.
\end{itemize}
    
\paragraph{Group}
The group is a collection of users and devices created by the organization.
Only users and devices belonging to the same group can interact with each other.
Users can create tunnels between devices in the same group.
The group has the following attributes:
\begin{itemize}
    \item name: The identifier of the group.
    \item users: A list of users.
    \item devices: A list of devices.
\end{itemize}
