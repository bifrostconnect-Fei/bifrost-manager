During the requirement analysis, we can base on the company's business domain 
to define the domain model.

The domain model is a conceptual model of the domain that incorporates both
behavioral and data concepts.
The domain model is used to describe the various entities, their attributes,
and their relationships.

The domain model following:

\paragraph{Organization}
The organization is the customer who will use the Bifrost unit (the product) and solution
to solve their remote access issues.

It has all the users and devices and can group them together to allow the users to
use the devices to connect to remote equipment.

The organization has the following attributes:
\begin{itemize}
    \item name: The name of the organization.
    \item users: A collection of users.
    \item devices: A collection of devices.
    \item groups: A list of groups.
\end{itemize}

\paragraph{User}
The user can be a technician or an operator who uses the device to connect to remote equipment
from the local equipment. They are the starting and ending point of all processes. 

The user also can be the manager who manage the users and devices in the organization.
Or can be the normal user who belongs to a group to use the device in same group to 
connect to remote equipment.

The user has the following attributes:
\begin{itemize}
    \item username: The identifier of the user.
    \item role: The role of the user, can be normal user or admin.
\end{itemize}

\paragraph{Device}
The device is the hardware that using for connects to remote equipment.
Which is the part of the product from BifrostConnect. The device can be set in
a group to allow the user to use it to connect to remote equipment.

The device also have two different types, one is Attend Unit, another is
Unattended Unit.

The device has the following attributes:
\begin{itemize}
    \item name: The identifier of the device.
    \item type: The type of the device, can be Attend Unit or Unattended Unit.
\end{itemize}
    
\paragraph{Group}
The group is a collection of users and devices created by the organization.
Only users and devices belonging to the same group can interact with each other.

Which means the user can only use the device in the same group to connect to remote equipment.

The group has the following attributes:
\begin{itemize}
    \item Name: The identifier of the group.
    \item Users: A list of users.
    \item Devices: A list of devices.
\end{itemize}
