% 这一部分是对商业领域模型的分析
% 在这个项目中, 根据项目需求的描述, 我们可以得到以下的商业领域模型
% 商业领域模型中包含以下的实体
% 用户, 设备, 组织, 小组, 
% 用户分为普通用户和管理员用户
% 设备分为 attend device 和 non-attend device
% 组织包含小组, 在这个项目中, 组织只有一个, 组织拥有所有设备和所有用户, 
% 管理员用户可以对设备和普通用户进行管理, 并且将设备和普通用户分配到小组中
% 普通用户只能使用所在小组的设备, 并且只能查看所在小组的普通用户


% 这一部分的逻辑是这样的是;
% 首先我们需要对商业领域模型进行分析, 然后根据商业领域模型进行对象领域模型的抽象, 
% 这个对象领域模型就是整个项目的领域模型基础, 
% 由此可以进行数据库模型的设计, 定义数据库中的表就是模型的实体, 表的字段就是模型的属性, 
% 同时根据关系, 指定表之间的关系, 定义私有键和外来键, 这样就可以得到数据库模型
% 然后还可以对后端对象的领域模型进行设计, 这个对象领域模型就是后端对象的基础, 

As we already discrbed about the bussines target and requirement 
about this project. We can identify the bussines domain model as following
entitis:

\begin{itemize}
    \item User
    \item Device
    \item Organization
    \item Group
\end{itemize}

\paragraph{User} is the entity that will use all the product and solution 
in the bussines domain. 

The User have the following attributes:

\begin{itemize}
    \item name : the name of the user
    \item email : the email of the user
    \item password : the password of the user
    \item role : the role of the user, can be admin or normal
    \item orgainzation : the organization that the user belongs to
    \item group : the group that the user belongs to
\end{itemize}

\paragraph{Device} is the entity that will be used by the user to access 
to the remote equipment.

The Device have the following attributes:

\begin{itemize}
    \item name : the name of the device
    \item ip : the ip address of the device
    \item port : the port of the device
    \item type : the type of the device, can be attend or non-attend
    \item organization : the organization that the device belongs to
    \item group : the group that the device belongs to
\end{itemize}



