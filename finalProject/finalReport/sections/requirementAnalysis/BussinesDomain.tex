% 这一部分是对商业领域模型的分析
% 在这个项目中, 根据项目需求的描述, 我们可以得到以下的商业领域模型
% 商业领域模型中包含以下的实体
% 用户, 设备, 组织, 小组, 
% 用户分为普通用户和管理员用户
% 设备分为 attend device 和 non-attend device
% 组织包含小组, 在这个项目中, 组织只有一个, 组织拥有所有设备和所有用户, 
% 管理员用户可以对设备和普通用户进行管理, 并且将设备和普通用户分配到小组中
% 普通用户只能使用所在小组的设备, 并且只能查看所在小组的普通用户


% 这一部分的逻辑是这样的是;
% 首先我们需要对商业领域模型进行分析, 然后根据商业领域模型进行对象领域模型的抽象, 
% 这个对象领域模型就是整个项目的领域模型基础, 
% 由此可以进行数据库模型的设计, 定义数据库中的表就是模型的实体, 表的字段就是模型的属性, 
% 同时根据关系, 指定表之间的关系, 定义私有键和外来键, 这样就可以得到数据库模型
% 然后还可以对后端对象的领域模型进行设计, 这个对象领域模型就是后端对象的基础, 

As we already discrbed about the bussines target and requirement 
about this project. We can identify the bussines domain model as following
entitis:

\paragraph{Organization}
The organization have all the users and devices, which can set up the users and devices as a group.
Therefore the organization have the following attributes:
\begin{itemize}
    \item name: The name of the organization.
    \item users: which will be a list of users.
    \item devices: which will be a list of devices.
    \item groups: which will be a list of groups.
\end{itemize}

\paragraph{User}
The user can be the technitian or the oprator who is using the device to connect
to the remote equipment. The user will be the end point which will start all of the process and end of the process as well.
The user have the following attributes:
\begin{itemize}
    \item username: which will identify the user.
\end{itemize}

\paragraph{Device}
The device is the hardware which will be connected to the remote equipment to allowed the user to control the remote equipment.
The device have the following attributes:
\begin{itemize}
    \item name: which will identity the device.
\end{itemize}
    
\paragraph{Group}
The group is the collection of the users and devices, which will be set up by the organization. 
Only the user and the device belong to the same group, the user can oprate the device, and only both the devices in the same group.
The user can create the tunnel between the devices in the same group.
The group have the following attributes:
\begin{itemize}
    \item name: which will identity the group.
    \item users: which will be a list of users.
    \item devices: which will be a list of devices.
\end{itemize}