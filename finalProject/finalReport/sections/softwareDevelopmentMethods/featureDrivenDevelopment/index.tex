Feature Driven Development (FDD) 
is an iterative and incremental software development process.

The FDD is a model-driven short-iteration process 
that consists of five basic activities:

\begin{enumerate}
    \item Develop an overall model
    \item Build a feature list
    \item Plan by feature
    \item Design by feature
    \item Build by feature
\end{enumerate}

\subsubsection{Develop an overall model}
The first step of the FDD is to develop an overall model. 
The overall model is a domain model that preposed to let the application
can have a basic structure and can be run for no feature.
After the overall model is developed, 
the feature list can be created for adding the features to the application.

In this project, because the main focus of this project is to migrate
the interfaces into the exist application, the overall model will be only 
fouce on the UI, state management, and constructure of interface
in the Device Manager.

\subsubsection{Feature list}
The feature list is a list of the features that the application will have.
It will following the use-case that be indentify in the user story. 
And connect the all front-end, back-end and database together.
Therefore, the each feature will produce the sequence start 
from the front-end to the back-end and then to the database, 
and return the result to the front-end to display to the user.