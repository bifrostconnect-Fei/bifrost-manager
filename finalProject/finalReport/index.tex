\documentclass{article}
\usepackage{graphicx}
\usepackage{amsmath}
\usepackage{tikz}

\usetikzlibrary{shapes,arrows}

% \usepackage{cite} % Add this line for citation

% \bibliographystyle{plain}

\title{The implementation of Bifrostconnect Front-end scope, re-design and
development with the relevant back-end support develop.}
\author{Fei Gu}
\date{\today}


\begin{document}
\maketitle
\section*{Abstract}
\pagebreak

\tableofcontents
\pagebreak

\section{Introduction}

    \subsection{The project type}
    % 关于项目的类型和背景的介绍
% 项目是 EASV computer science 专业的毕业设计, 毕业设计的要求是充分的展现学生在两年半的学习中
% 能够获得的知识和技能, 并且能够在一个项目中充分的展现出来. 项目的类型是一个软件开发项目,
% 项目的背景是一个公司的项目, 项目的目的是为了解决公司的一个问题.

This project is the final project of the EASV computer science program. 
The purpose of the final project is to demonstrate the knowledge 
and skills by the student during the two and a half years of study,
and be able to fully demonstrate them in a project.

In this project, the student will work with a company 
to solve a problem of the company.
The student will use the knowledge and skills learned from the education
to solve the problem.

Which means this project will relavent to the company's business domain,
and using the software engineering methods and software development methods
to reconize the problem, analyze the problem, and solve the problem.



    \subsection{About the company}
    % 关于公司的介绍
% 公司的名字是 BifrostConnect, 是一个创业公司, 公司的主要业务是为了解决在严格的系统安全条件下, 
% 当客户的系统没有也不允许有互联网的情况下, 通过一个中继设备而不是第三方服务的条件下, 能够远程的
% 访问客户的系统. 公司的产品是一个硬件设备以及配套的网络应用. 以及相关的服务和一站式解决方案.

BifrostConnect is a startup company which established in 2018 
that has 10 to 12 employees and located in Copenhagen, Denmark.

During the five years of operation, the company has developed
a remote access solution for access the users remote system.
The solution  


The company's main business is to solve the problem of remote access 
to the customer's system when the customer's system does not have 
and is not allowed to have the Internet under strict system security conditions.

The company's product is a hardware device and a network application.
And related services and one-stop solutions.

    \subsection{About the project}
    % 关于项目的介绍
% 这个项目是在公司原有软件基础上, 对现有的前端软件进行重构和重新设计, 并且在后端提供相应的支持.

% This project is to re-design and re-develop the front-end software based on the company's original software,
% and provide corresponding support on the back-end.% Introduction to the project


% This project involves a comprehensive overhaul of the existing front-end software, building upon the company's original software. It also includes providing corresponding back-end support.

This project entails a thorough redesign and redevelopment of the existing front-end software, 
leveraging the foundation provided by the company's original software. 
Additionally, it involves enhancing the back-end to provide the necessary support 
for the new front-end.

\pagebreak

\section{Problem Statement}
    
        \subsection{Statement}
        During this project the Remote Access Interface and Tunnel Interface should be integrated into the Device Mananger Application.
To achieve the one single page application with full functionality of the Remote Access and Tunnel Connection capabilities.

    
        \subsection{Situation}
        % 项目的背景
% 这一部分要详细的介绍关于项目的原始背景, 因为在之后要根据这些背景写明项目的意义和解决方案.
% 所以项目的现状, 项目的历史问题, 现有项目难以开发和维护的原因, 

The current BifrostConnect front-end application consist of the Device Manager (DM) 
and Remote Access Interface (RAI) two different user interface.
Which means the user need to manage their devices and users in the one application, 
and oprate their devices to use the bifrost product in other web application as well. 
However, the RAI consists of two different user interfaces referred to as the Classic RAI and Tunnel RAI. 
Those two user interfaces will leading the user to different web application 
when user wish to create a different type of access solution.

The reason for this situation occurred is because the bifrost product is developed focus on the device 
and the application part was not considered as important as the device part. 
Therefore, the application part was developed by different developers with different design logic.
And the application functionality was not been properly through out the development process. 
Which cause the design logic of this product is not scalable, leading to decreasing user experience, 
and exponential growth of code complexity which makes it increasingly challenging to maintain and improve. 
Also the product design style lacks scalability due to lack of modularity and code consistency, 
and the operation logic is not clear.
    
        \subsection{Potential Solution}
        To address those issue, there is a need to redesign and remake the front-end with
a focus on implementing a modular unified front-end management system that
encompasses all the necessary functionality. 

The goal is to provide users with seamless control, monitoring, and access to their devices from anywhere, 
enhancing their overall user experience and productivity while keeping code complexity as low as possible.

For the implementation to be successful, we also need to dedicate a back-end
service, including an API and Database. The back-end server will contain the
necessary service for the front-end to reach the purpose. Such as the RESTful
API service, authentication service, MQTT service, Data  Channel service and Database.


        \pagebreak

\section{Requirement Analysis}
The continuous integration is the practice of merging the code from each of develop process 
into a source control repository to integrate and test the code.

The proposal for continuous integration is to quickly check the code is working for different part
of the project together which means integration.

For this project, each of the feature will be developed in separate branch and should be merged into
the master branch and test as a whole project.

Therefore, when the feature is developed, the developer should push the code to the remote repository, 
and create a pull request to merge the code into the master branch.

\paragraph{Pull Request}

When the developer pushes the code to the remote repository, the pull request will be created automatically.

The pull request will trigger the continuous integration process, which will build the project and run the test cases.
If the test cases are passed, the pull request will be approved and the code will be merged into the master branch.

The continuous integration process will be implemented by the GitHub Actions.

Start from the checkout the code from the remote repository, 
the GitHub Actions will build the project.

For the front-end project, automatically the GitHub Actions will set up the Node.js environment,
install the dependencies, and build the project using the NPM command.

After build the project, the GitHub Actions will 

\paragraph{Continuous Testing}

The continuous testing is the practice of running automated test cases by using the   NPM command in the GitHub Actions.

That will lead the NPM to run the unit test cases and then generate the test coverage report.

After the unit test, the GitHub Actions will run the end-to-end test cases by using the E2E testing framework.


        \subsection{Stakeholders}
        The continuous integration is the practice of merging the code from each of develop process 
into a source control repository to integrate and test the code.

The proposal for continuous integration is to quickly check the code is working for different part
of the project together which means integration.

For this project, each of the feature will be developed in separate branch and should be merged into
the master branch and test as a whole project.

Therefore, when the feature is developed, the developer should push the code to the remote repository, 
and create a pull request to merge the code into the master branch.

\paragraph{Pull Request}

When the developer pushes the code to the remote repository, the pull request will be created automatically.

The pull request will trigger the continuous integration process, which will build the project and run the test cases.
If the test cases are passed, the pull request will be approved and the code will be merged into the master branch.

The continuous integration process will be implemented by the GitHub Actions.

Start from the checkout the code from the remote repository, 
the GitHub Actions will build the project.

For the front-end project, automatically the GitHub Actions will set up the Node.js environment,
install the dependencies, and build the project using the NPM command.

After build the project, the GitHub Actions will 

\paragraph{Continuous Testing}

The continuous testing is the practice of running automated test cases by using the   NPM command in the GitHub Actions.

That will lead the NPM to run the unit test cases and then generate the test coverage report.

After the unit test, the GitHub Actions will run the end-to-end test cases by using the E2E testing framework.


        \subsection{Bussines Domain}
        The continuous integration is the practice of merging the code from each of develop process 
into a source control repository to integrate and test the code.

The proposal for continuous integration is to quickly check the code is working for different part
of the project together which means integration.

For this project, each of the feature will be developed in separate branch and should be merged into
the master branch and test as a whole project.

Therefore, when the feature is developed, the developer should push the code to the remote repository, 
and create a pull request to merge the code into the master branch.

\paragraph{Pull Request}

When the developer pushes the code to the remote repository, the pull request will be created automatically.

The pull request will trigger the continuous integration process, which will build the project and run the test cases.
If the test cases are passed, the pull request will be approved and the code will be merged into the master branch.

The continuous integration process will be implemented by the GitHub Actions.

Start from the checkout the code from the remote repository, 
the GitHub Actions will build the project.

For the front-end project, automatically the GitHub Actions will set up the Node.js environment,
install the dependencies, and build the project using the NPM command.

After build the project, the GitHub Actions will 

\paragraph{Continuous Testing}

The continuous testing is the practice of running automated test cases by using the   NPM command in the GitHub Actions.

That will lead the NPM to run the unit test cases and then generate the test coverage report.

After the unit test, the GitHub Actions will run the end-to-end test cases by using the E2E testing framework.


        \subsection{Scope}
        % The scope of this project is to develop a web application 
% which can be used to manage the device and users, 
% and create the RAI and Tunnel.

% Because the device manager already exist, 
% the mean focus will be on the implementation of the RAI and Tunnel into the device manager.
% Which means we don't need to develop the device manager, 
% also there are no necceary to develop the new features for the RAI and Tunnel interface.

% Because the situation of the entire system was redudent and coupling, 
% we have to re-design the system components and identify the relationship between them.
% The re-design of the system will be the main focus of this project.

% Since the Device Manager is written in React framework, 
% we have to use the same framework to develop the RAI and Tunnel interface.

The scope of this project is to develop a web application 
that manages devices and users, and creates the RAI and Tunnel.

Since the device manager already exists, 
the main focus will be on implementing the RAI and Tunnel 
into the device manager.

Therefore, there is no need to develop the device manager 
or add new features to the RAI and Tunnel interface.

Due to the redundant and coupled nature of the entire system, 
we need to redesign the system components 
and establish their relationships.

As the Device Manager is built using the React framework, 
we will also use the same framework to develop the RAI and Tunnel interface.

The main focus of this project is to migramt the RAI and Tunnel interface, 
and re-design the UI, components, and the relavent methodologies.

    
        \subsection{Goals}
        The goal of this project is to develop a RAI and Tunnel interface 
in the Device Manager that can create, modify, delete, 
and view access connections. 
Additionally, it should have the capability 
to remotely control the device and target equipment.


        \pagebreak

\section{Software Design}

    \subsection{Software Development Methods}
    The continuous integration is the practice of merging the code from each of develop process 
into a source control repository to integrate and test the code.

The proposal for continuous integration is to quickly check the code is working for different part
of the project together which means integration.

For this project, each of the feature will be developed in separate branch and should be merged into
the master branch and test as a whole project.

Therefore, when the feature is developed, the developer should push the code to the remote repository, 
and create a pull request to merge the code into the master branch.

\paragraph{Pull Request}

When the developer pushes the code to the remote repository, the pull request will be created automatically.

The pull request will trigger the continuous integration process, which will build the project and run the test cases.
If the test cases are passed, the pull request will be approved and the code will be merged into the master branch.

The continuous integration process will be implemented by the GitHub Actions.

Start from the checkout the code from the remote repository, 
the GitHub Actions will build the project.

For the front-end project, automatically the GitHub Actions will set up the Node.js environment,
install the dependencies, and build the project using the NPM command.

After build the project, the GitHub Actions will 

\paragraph{Continuous Testing}

The continuous testing is the practice of running automated test cases by using the   NPM command in the GitHub Actions.

That will lead the NPM to run the unit test cases and then generate the test coverage report.

After the unit test, the GitHub Actions will run the end-to-end test cases by using the E2E testing framework.


        \subsubsection{Agile Software Development}
        The continuous integration is the practice of merging the code from each of develop process 
into a source control repository to integrate and test the code.

The proposal for continuous integration is to quickly check the code is working for different part
of the project together which means integration.

For this project, each of the feature will be developed in separate branch and should be merged into
the master branch and test as a whole project.

Therefore, when the feature is developed, the developer should push the code to the remote repository, 
and create a pull request to merge the code into the master branch.

\paragraph{Pull Request}

When the developer pushes the code to the remote repository, the pull request will be created automatically.

The pull request will trigger the continuous integration process, which will build the project and run the test cases.
If the test cases are passed, the pull request will be approved and the code will be merged into the master branch.

The continuous integration process will be implemented by the GitHub Actions.

Start from the checkout the code from the remote repository, 
the GitHub Actions will build the project.

For the front-end project, automatically the GitHub Actions will set up the Node.js environment,
install the dependencies, and build the project using the NPM command.

After build the project, the GitHub Actions will 

\paragraph{Continuous Testing}

The continuous testing is the practice of running automated test cases by using the   NPM command in the GitHub Actions.

That will lead the NPM to run the unit test cases and then generate the test coverage report.

After the unit test, the GitHub Actions will run the end-to-end test cases by using the E2E testing framework.


        \subsubsection{Feature Driven Development}
        The continuous integration is the practice of merging the code from each of develop process 
into a source control repository to integrate and test the code.

The proposal for continuous integration is to quickly check the code is working for different part
of the project together which means integration.

For this project, each of the feature will be developed in separate branch and should be merged into
the master branch and test as a whole project.

Therefore, when the feature is developed, the developer should push the code to the remote repository, 
and create a pull request to merge the code into the master branch.

\paragraph{Pull Request}

When the developer pushes the code to the remote repository, the pull request will be created automatically.

The pull request will trigger the continuous integration process, which will build the project and run the test cases.
If the test cases are passed, the pull request will be approved and the code will be merged into the master branch.

The continuous integration process will be implemented by the GitHub Actions.

Start from the checkout the code from the remote repository, 
the GitHub Actions will build the project.

For the front-end project, automatically the GitHub Actions will set up the Node.js environment,
install the dependencies, and build the project using the NPM command.

After build the project, the GitHub Actions will 

\paragraph{Continuous Testing}

The continuous testing is the practice of running automated test cases by using the   NPM command in the GitHub Actions.

That will lead the NPM to run the unit test cases and then generate the test coverage report.

After the unit test, the GitHub Actions will run the end-to-end test cases by using the E2E testing framework.


    \subsection{Technology selection}
    The continuous integration is the practice of merging the code from each of develop process 
into a source control repository to integrate and test the code.

The proposal for continuous integration is to quickly check the code is working for different part
of the project together which means integration.

For this project, each of the feature will be developed in separate branch and should be merged into
the master branch and test as a whole project.

Therefore, when the feature is developed, the developer should push the code to the remote repository, 
and create a pull request to merge the code into the master branch.

\paragraph{Pull Request}

When the developer pushes the code to the remote repository, the pull request will be created automatically.

The pull request will trigger the continuous integration process, which will build the project and run the test cases.
If the test cases are passed, the pull request will be approved and the code will be merged into the master branch.

The continuous integration process will be implemented by the GitHub Actions.

Start from the checkout the code from the remote repository, 
the GitHub Actions will build the project.

For the front-end project, automatically the GitHub Actions will set up the Node.js environment,
install the dependencies, and build the project using the NPM command.

After build the project, the GitHub Actions will 

\paragraph{Continuous Testing}

The continuous testing is the practice of running automated test cases by using the   NPM command in the GitHub Actions.

That will lead the NPM to run the unit test cases and then generate the test coverage report.

After the unit test, the GitHub Actions will run the end-to-end test cases by using the E2E testing framework.


            \subsubsection{Front-end}
            \paragraph{JavaScript}
The JavaScript is a fundamental programming language of web development.
Since the original front-end application was using this, there are no reason to change it.

\paragraph{React} 
The React is a JavaScript library for building user interfaces. 
It is maintained by Facebook and a community of individual developers and companies.
React can be used as a base in the development of single-page or mobile applications.

\paragraph{React-Router} 
The React-Router is a routing library for React.
It abstracts away the details of server and client and allows developers 
to focus on building apps.

\paragraph{Redux}
The Redux is a predictable state container for JavaScript apps.
It helps you write applications that behave consistently, 
run in different environments (client, server, and native),and are easy to test.
On top of that, it provides a great developer experience, 
such as live code editing combined with a time traveling debugger.

\paragraph{RTK} 
The RTK is a powerful, opinionated Redux tool set for writing better reducers,
organizing and accessing state in components, and managing side effects.

\paragraph{RTKQ}
RTKQ is a powerful data fetching and caching tool.
It is designed to simplify common cases for loading data in a web application, 
eliminating the need to hand-write data fetching \& caching logic yourself.


\paragraph{Tailwind CSS} 
The Tailwind CSS is a utility-first CSS framework for rapidly building custom user interfaces.
It is a highly customizable, low-level CSS framework that gives you all the building blocks 
you need to build bespoke designs without any annoying opinionated styles you have to fight to override.


\paragraph{Semantic UI} is a development framework that helps create beautiful, responsive layouts using human-friendly HTML.
Semantic UI treats words and classes as exchangeable concepts.
Classes use syntax from natural languages like noun/modifier relationships, word order, and plurality to link concepts intuitively.


\paragraph{Jest} 
The Jest is a JavaScript testing framework maintained by Facebook, Inc.
designed and built by Christoph Nakazawa with a focus on simplicity and support for large web applications.
It works with projects using Babel, TypeScript, Node.js, React, Angular, Vue.js and Svelte.
Jest does not require a browser to run tests, and it runs tests in parallel - this makes Jest fast.
Jest is well-documented, requires little configuration and can be extended to match your requirements.

            
            \subsubsection{Back-end}
            \paragraph{CSharp}
CSharp is a general-purpose, multi-paradigm programming language.

\paragraph{dotNET Core}
dotNET is a free and open-source, managed computer software framework for Windows, Linux, and macOS operating systems.

\paragraph{Entity Framework}
EF is an open-source ORM framework for .NET applications supported by Microsoft.

\paragraph{xUnit}
xUnit is a unit testing framework for the .NET.

\paragraph{Moq}
Moq is a mocking library for .NET for mocking interfaces and classes.

            
            \subsubsection{Database}
            \input{sections/softwareDesign/technologySelection/database}
            
            \subsubsection{Data communication}
            \paragraph{HTTP}
is an application-layer protocol for transmitting hypermedia documents, such as HTML.
It was designed for communication between web browsers and web servers, but it can also be used for other purposes.
HTTP follows a classical client-server model, with a client opening a connection to make a request,
then waiting until it receives a response.
HTTP is a stateless protocol, meaning that the server does not keep any data (state) between two requests.
Though often based on a TCP/IP layer, it can be used on any reliable transport layer;
that is, a protocol that doesn't lose messages silently, such as UDP.

\paragraph{Flatbuffers}
is an efficient cross platform serialization library for C++, C\#, C, Go, Java, JavaScript, Lobster, Lua, TypeScript, PHP, Python, and Rust.
It was originally created at Google for game development and other performance-critical applications.

\paragraph{MQTT}
is a machine-to-machine (M2M)/"Internet of Things" connectivity protocol.
It was designed as an extremely lightweight publish/subscribe messaging transport.
It is useful for connections with remote locations where a small code footprint is required and/or network bandwidth is at a premium.

\paragraph{webrtc}
is a free, open-source project that provides web browsers and mobile applications with real-time communication (RTC) via simple application programming interfaces (APIs).
It allows audio and video communication to work inside web pages by allowing direct peer-to-peer communication,
eliminating the need to install plugins or download native apps.

\paragraph{WebRTC Datachannel}
is a component of WebRTC that enables peer-to-peer exchange of arbitrary data between two devices.
It was added to the WebRTC standard by the World Wide Web Consortium (W3C) in January 2015.

            
            \subsubsection{DevOps}
            \paragraph{Git}
Git is a distributed version-control system for tracking changes in source code during software development.
It is designed for coordinating work among programmers, but it can be used to track changes in any set of files.
Its goals include speed, data integrity, and support for distributed, non-linear workflows.

\paragraph{GitHub}
GitHub is a provider of Internet hosting for software development and version control using Git.
It offers the distributed version control and source code management (SCM) functionality of Git,
plus its own features.
It provides access control and several collaboration features such as bug tracking, feature requests, task management, and wikis for every project.

\paragraph{GitHub Actions}
GitHub Actions is an API for cause and effect on GitHub: orchestrate any workflow, based on any event, 
while GitHub manages the execution, provides rich feedback, and secures every step along the way.
With GitHub Actions, workflows and steps are just code in a repository, so you can create, share, reuse, and fork your software development practices.

\paragraph{Docker}
Docker is a set of platform as a service (PaaS) products that use OS-level virtualization to deliver software in packages called containers.
Containers are isolated from one another and bundle their own software, libraries and configuration files;
they can communicate with each other through well-defined channels.
All containers are run by a single operating system kernel and are thus more lightweight than virtual machines.
Containers are created from images that specify their precise contents.
Images are often created by combining and modifying standard images downloaded from public repositories.


    \subsection{Architecture design} % here should be rewrite later
    The continuous integration is the practice of merging the code from each of develop process 
into a source control repository to integrate and test the code.

The proposal for continuous integration is to quickly check the code is working for different part
of the project together which means integration.

For this project, each of the feature will be developed in separate branch and should be merged into
the master branch and test as a whole project.

Therefore, when the feature is developed, the developer should push the code to the remote repository, 
and create a pull request to merge the code into the master branch.

\paragraph{Pull Request}

When the developer pushes the code to the remote repository, the pull request will be created automatically.

The pull request will trigger the continuous integration process, which will build the project and run the test cases.
If the test cases are passed, the pull request will be approved and the code will be merged into the master branch.

The continuous integration process will be implemented by the GitHub Actions.

Start from the checkout the code from the remote repository, 
the GitHub Actions will build the project.

For the front-end project, automatically the GitHub Actions will set up the Node.js environment,
install the dependencies, and build the project using the NPM command.

After build the project, the GitHub Actions will 

\paragraph{Continuous Testing}

The continuous testing is the practice of running automated test cases by using the   NPM command in the GitHub Actions.

That will lead the NPM to run the unit test cases and then generate the test coverage report.

After the unit test, the GitHub Actions will run the end-to-end test cases by using the E2E testing framework.


        \subsubsection{Front-end}
        In the architecture design of the front-end, the primary goal is to migrate the existing 
remote access interface and the tunnel interface into the web application.
And the secondary goal is refactoring the device manager as the well structured web application.

The front-end utilizes the React framework based on view design and the Redux framework 
based on state and data flow.

The structure of the React framework divides the web page into multiple components 
based on modularity.Each component has its own state and properties.

Components are categorized into container components, presentational components, 
and functional components.

Container components manage data and state, presentational components display data 
and state, and functional components implement specific functionalities.

As beginning of this project, the migrate two interfaces which was use the vanilla JS, 
have to be redesigned and refectoring to the React component.

\paragraph{Remote Access Interface}

The remote access interface need to be redesigned as a single page application, 
and its perpose is to provide the interface to create the reomote access connection, 
display the connection status, the device status, and the data stream from the device.

For the UI of the remote access interface, the component will be inject into the RAI page 
from the page component, and the component will be the bifrost page component which is 
the reusable page component directory.  
the pototype descibed following:

\begin{itemize}
    \item The top of the component is the status bar, which contains the status of the device.
    \begin{itemize}
        \item The left of the status bar is the device name, serial number, and device framework version.
        \item The middle of the status bar is the device status, it will display the device status, 
        such like charging status, battery status, and the device connection status.
        \item The right of the status bar is the exit session button, it will exit the current session.
    \end{itemize}
    \item The main content is the body of the component, which display the data stream from the device.
    \item The bottom of the compoent is the control bar, it includes two part.
    \begin{itemize}
        \item The left of the control bar is the tag swither, it will switch the data stream between the
        KVM data stream, and the serial data stream.
        \item The right of the control bar is the control buttons, it include the setting button, 
        will display the setting menu, the full screen button, will display the data stream in full screen
        in the browser, and keymap button, will display the keymap menu.
    \end{itemize}
\end{itemize}

The UI state will be isolated from the session state which will be interactive with the session state.
For UI the state will not store in the redux store, it will store in each of the component.

For the session be create for this interface, it will be store in the redux store as the session state.
The session state will be store in the session reducer, and the session reducer will be combine with
the root reducer. 

The session state will be store as the object, and the object will be store in the session array.
Use this pattern we can easily to create mutiple session for the remote access interface, and easily to 
manage, switch, and delete the session.

Each session will have its own state, the state include:
\begin{itemize}
    \item id, which is the unique id for the session. Because each of the device can only
    active one session at the same time, so the session id will be the same as the device serial number.
    \item name, which is the name of the session, we can use the device name 
    as the session name.
    \item status, this is the flag of the session active or not.
    \item WebRTC connection, which is the WebRTC connection object of the session.
    \item MQTT connection, which is the MQTT connection object of the session.
\end{itemize}

For create the session , there are some hooks will be use in the component, for example:
\begin{itemize}
    \item useCreateSession, this hook will be use to create the session, 
    and it will return the session object.
    \item useCloseSession, this hook will be use to close the session, 
    and it will return the session object.
    \item keepAliveSession, this hook will be use to keep the session alive, 
    and it will return the session object.
    \item useMQTT, this hook will be use to subscribe the MQTT topic, receive the MQTT message, 
    and publish the MQTT message.
    \item useWebRTC, this hook will be use to create the WebRTC connection, and so on.
    \item useFlatBuffer, this hook will be use to process the flatbuffer message.
\end{itemize}

\paragraph{Tunnel Interface}
Same as the remote access interface, the tunnel interface also need to be redesigned  
as a single page application,
and its perpose is to provide the interface to create the tunnel connection, display the connection status,
the device status, and the data stream from the device.

For the UI of the tunnel interface, it will be following the same pattern as the remote access interface,
and the UI design will be little different.

The reason fot the different UI design is the tunnel interface will be use two device to create the tunnel,
connecting each of the end point though the tunnel.

The UI will be scribed following:
\begin{itemize}
    \item The top of the component is the connecting status bar, which contains the each of the device status.
    Between the two device status will be the tunnel status. 
    \item The main content is the body of the component, excipt to display the data stream from the device, 
    it also include a sidebar for display the IP mapping list.
\end{itemize}

The rest of the state design and the hooks design will be the same as the remote access interface 
depends on the reuse of the component. 

        \subsubsection{Back-end}
        The backend uses the dotNet framework. The development language using the C\# language.

In this project, the backend uses the Onion Architecture.
The Onion Architecture is a typically layered architecture, 
where each layer depends on the inner layer and provides interfaces to the outer layer.
The outer layer provides services to the outermost layer 
and other modules in the same layer based on the interfaces of the inner layer.

From inner to outer, the layers are: Domain, Application, Infrastructure, Presentation.
The Domain layer is the core layer and the innermost layer, used to define domain models, 
which are the business models.
It includes domain models and domain service interfaces.
Domain models are used to define the business models, 
which are the entities in the entity-relationship model and their attributes.
Domain service interfaces are used to define the business services, 
which are the relationships between entities in the entity-relationship model.

The Application layer is the application layer, 
used to define application services, which are the business logic.
It includes domain service implementations and application service interfaces.
Domain service implementations implement the methods of the inner layer's domain service 
interfaces and implement the business logic of the domain models.
Application service interfaces are used to define application services, 
which are the business logic.
It includes but is not limited to database interfaces, testing interfaces, 
HTTP API interfaces, MQTT interfaces, etc.

The Infrastructure layer is the infrastructure layer, used to define infrastructure.
It includes database implementations, testing implementations, 
HTTP API implementations, MQTT implementations, etc.
Database implementations implement the database interfaces 
and provide CRUD services for the database.
Testing implementations implement the testing interfaces 
and provide services for unit testing and integration testing.
HTTP API implementations implement the HTTP API interfaces 
and provide CRUD operations for HTTP APIs.
MQTT implementations implement the MQTT interfaces 
and provide CRUD operations for MQTT.

The Presentation layer is the presentation layer, used to define presentation logic, 
such as interfaces and pages. Since this is a backend project,
data presentation and control are handled by the frontend, 
so this layer is not needed.




        \subsubsection{Data communication and storage}
        % Design of data communication and data storage for this project, 
% including the selection of communication protocols and the design of data storage.
Design of data communication:
1. Selection of communication protocols
There are three types of data transmitted from the front-end to the back-end: 
- Requests for basic CRUD operations, which have low requirements for timeliness but require accuracy, integrity, order, and security. 
For this type of data transmission, the HTTP protocol and RESTful API design are used to effectively meet the above requirements.
- Creation of data channels and transmission of streaming media data, which require high timeliness and security. 
For this type of data transmission, the WebRTC protocol and MQTT protocol are used, along with the fast-decoding flatbuffers protocol, to effectively meet the above requirements.
- Transmission of device status information and operation information, which require high integrity, order, and security. 
For this type of data transmission, the MQTT protocol is used along with the flatbuffers protocol.

2. Communication architecture and flow of data communication
The data communication architecture of this project is based on a front-end and back-end separation architecture, with React framework used for the front-end and dotnet framework used for the back-end.
When the front-end needs to send data to the back-end, it sends an HTTP request to the back-end. Upon receiving the HTTP request, the back-end selects different data processing methods based on the data type of the request. 
For basic CRUD requests, the back-end performs CRUD operations on the data based on the RESTful API design. 
For the creation of data channels and transmission of streaming media data, the back-end creates data channels based on the WebRTC protocol and helps establish data channels between the front-end and devices. Once the data channel is established, the front-end and devices use the flatbuffer data format for streaming media data transmission.
For the transmission of device status information and operation information, the front-end directly sends MQTT requests to the MQTT broker. The devices listen to the relevant MQTT requests in their firmware and return the corresponding data.

3. Data formats for data communication
There are three data formats for data communication in this project:
- HTTP protocol: Data is transmitted using the JSON format.
- WebRTC protocol: Data is transmitted using the flatbuffers format.
- MQTT protocol: Data is transmitted using the flatbuffers format.

Design of data storage:
1. Selection of database for data storage
For data storage in this project, a lightweight database SQLite is used.
SQLite is an in-process library that implements a self-contained, serverless, zero-configuration, transactional SQL database engine.
This choice is made because the purpose of the entire project is to achieve data communication between the front-end and devices. There is no high requirement for database operations such as CRUD, the data volume is small, and the transactional requirement for database data is not high. Therefore, SQLite database is chosen.

2. Design of data structures for the front-end and back-end of the project
In this project, the front-end uses the React framework, so the design of data structures for the front-end adopts a state-based approach.
Each component or data set contains a state object, and the properties of this state object represent the various states of the component.
The use of state objects facilitates state management. By using the object-property form, it is easy to distinguish between similar states of different components.
Since cross-component states are managed by Redux, this design of state objects enables easier state updates and transfers.
The back-end uses the dotnet framework, so the design of data structures for the back-end adopts a class-based approach.
Object-oriented programming principles are used to encapsulate data, making data transmission more secure, ordered, and complete.



    \subsection{Domain model}
    The continuous integration is the practice of merging the code from each of develop process 
into a source control repository to integrate and test the code.

The proposal for continuous integration is to quickly check the code is working for different part
of the project together which means integration.

For this project, each of the feature will be developed in separate branch and should be merged into
the master branch and test as a whole project.

Therefore, when the feature is developed, the developer should push the code to the remote repository, 
and create a pull request to merge the code into the master branch.

\paragraph{Pull Request}

When the developer pushes the code to the remote repository, the pull request will be created automatically.

The pull request will trigger the continuous integration process, which will build the project and run the test cases.
If the test cases are passed, the pull request will be approved and the code will be merged into the master branch.

The continuous integration process will be implemented by the GitHub Actions.

Start from the checkout the code from the remote repository, 
the GitHub Actions will build the project.

For the front-end project, automatically the GitHub Actions will set up the Node.js environment,
install the dependencies, and build the project using the NPM command.

After build the project, the GitHub Actions will 

\paragraph{Continuous Testing}

The continuous testing is the practice of running automated test cases by using the   NPM command in the GitHub Actions.

That will lead the NPM to run the unit test cases and then generate the test coverage report.

After the unit test, the GitHub Actions will run the end-to-end test cases by using the E2E testing framework.


    % \subsection{Database design}
    % % 数据库领域模型 ER 图
    % % 包括表和字段的设置.
    % % 对于私有键和外键的设置.

    % \subsection{Back-end design}
    % % 后端对象模型
    % % 以及对于对象模型的增删改查
    % % 以及相关的其他服务的设计`'

    % \subsection{Front-end design}
    % % 对于前端的页面结构的设计 
    % % 页面的状态的设计, 交互设计

    % \subsection{FlatBuffers design}
    % % schema 的设计

    \subsection{DevOps CI/CD process design}
    The continuous integration is the practice of merging the code from each of develop process 
into a source control repository to integrate and test the code.

The proposal for continuous integration is to quickly check the code is working for different part
of the project together which means integration.

For this project, each of the feature will be developed in separate branch and should be merged into
the master branch and test as a whole project.

Therefore, when the feature is developed, the developer should push the code to the remote repository, 
and create a pull request to merge the code into the master branch.

\paragraph{Pull Request}

When the developer pushes the code to the remote repository, the pull request will be created automatically.

The pull request will trigger the continuous integration process, which will build the project and run the test cases.
If the test cases are passed, the pull request will be approved and the code will be merged into the master branch.

The continuous integration process will be implemented by the GitHub Actions.

Start from the checkout the code from the remote repository, 
the GitHub Actions will build the project.

For the front-end project, automatically the GitHub Actions will set up the Node.js environment,
install the dependencies, and build the project using the NPM command.

After build the project, the GitHub Actions will 

\paragraph{Continuous Testing}

The continuous testing is the practice of running automated test cases by using the   NPM command in the GitHub Actions.

That will lead the NPM to run the unit test cases and then generate the test coverage report.

After the unit test, the GitHub Actions will run the end-to-end test cases by using the E2E testing framework.

        \subsubsection{Continuous Integration}
        % 持续集成
        % 详细的介绍持续集成, 包括持续集成的工具, 持续集成的流程, 持续集成的部署.

        \subsubsection{Continuous Delivery}
        % 持续交付
        % 详细的介绍持续交付, 包括持续交付的工具, 持续交付的流程, 持续交付的部署.

        \subsubsection{Continuous Deployment}
        % 持续部署
        % 详细的介绍持续部署, 包括持续部署的工具, 持续部署的流程, 持续部署的部署.

\pagebreak

\section{Software Development} 

    \subsection{Front-end}
    % 前端开发
    % 详细的介绍前端的开发, 包括前端的开发环境, 前端的开发工具, 前端的开发语言, 前端的开发框架, 前端的开发流程, 前端的开发测试, 前端的开发部署.

    \subsection{Back-end}
    % 后端开发
    % 详细的介绍后端的开发, 包括后端的开发环境, 后端的开发工具, 后端的开发语言, 后端的开发框架, 后端的开发流程, 后端的开发测试, 后端的开发部署.

    \subsection{Database}

    \subsection{WebRTC}

    \subsection{MQTT}

    \subsection{RESTful API}

    \subsection{Authentication}

\pagebreak

\section{Conclusion} 

    \subsection{Result}
    % 项目的结果
    % 项目的结果是什么, 项目的结果是否达到了预期的目标, 项目的结果是否达到了预期的效果, 项目的结果是否达到了预期的质量.

    \subsection{Discussion}
    % 项目的讨论
    % 项目的讨论是什么, 项目的讨论是对于项目的结果的讨论, 项目的讨论是对于项目的结果的分析, 项目的讨论是对于项目的结果的评价.

    \subsection{Future Work}
    % 项目的未来工作
    % 项目的未来工作是什么, 项目的未来工作是对于项目的结果的未来工作, 项目的未来工作是对于项目的结果的未来工作的分析, 项目的未来工作是对于项目的结果的未来工作的评价.


\pagebreak

% \bibliography{bibliography}

\pagebreak

% \begin{appendices}
%     \section{Appendix}
% \end{appendices} 

\end{document}